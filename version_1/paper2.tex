\documentclass[11pt, a4paper]{article}
\usepackage[utf8]{inputenc}
\usepackage{amsmath, amssymb, amsthm}
\usepackage{graphicx}
\usepackage{listings}
\usepackage{xcolor}
\usepackage{geometry}
\geometry{a4paper, margin=1in}
\usepackage{hyperref}
\usepackage{booktabs}

% Code listing style
\lstset{
    language=Python,
    basicstyle=\ttfamily\footnotesize,
    numbers=left,
    numberstyle=\tiny\color{gray},
    frame=single,
    breaklines=true,
    captionpos=b,
    tabsize=4,
    showspaces=false,
    showstringspaces=false,
    commentstyle=\color{gray},
    keywordstyle=\color{blue},
    stringstyle=\color{green!60!black}
}

% Define colors for better presentation
\definecolor{codegray}{gray}{0.9}
\definecolor{darkgreen}{rgb}{0.0, 0.5, 0.0}

\title{Quantum-Inspired Dynamic Phase Space Language Model: A Modular Approach}

\author{
    Gowrav Vishwakarma\thanks{Corresponding author: gowravvishwakarma@gmail.com}\\
    \textit{IT Department}\\
    \textit{Xavoc Technocrats Pvt. Ltd.}\\
    \textit{Udaipur, India}
}

\date{}

\begin{document}

\maketitle

\begin{abstract}
This paper introduces a novel language modeling framework inspired by quantum mechanics, designed to address key limitations of traditional neural network approaches. Our model employs dynamic phase spaces to represent language, wherein words are not static vectors, but rather multi-faceted entities with interwoven meaning, context, and emotional layers. The model utilizes universal constants to create stable phase spaces and leverages interference-based attention for information processing. We also propose a concept-based tokenizer that supports cross-lingual learning. The model is trained using unitary-inspired loss functions and coherence maximization techniques for stability. We provide a modular implementation, designed for practical application on current hardware, along with thorough mathematical justification for each component. This work seeks to establish a robust foundation for more adaptive, context-aware, and computationally efficient language models.
\end{abstract}

\section{Introduction}

Large language models (LLMs) have demonstrated exceptional abilities in natural language processing. However, they are constrained by fixed-dimensional vector representations, localized attention mechanisms, and limited contextual fluidity. This paper presents a paradigm shift in language modeling, drawing inspiration from the principles of quantum mechanics. Our model introduces a dynamic phase space where linguistic features are represented as multi-dimensional entities with interwoven meaning, context, and emotional layers. We leverage interference-based attention for capturing complex semantic relationships and adopt a concept-based tokenizer designed to support cross-lingual learning. The system is trained using unitary-inspired loss functions and coherence maximization to enhance stability and is implemented in a modular fashion for practical use on standard hardware. This approach provides a mathematically robust and practically viable path toward language models that are more adaptive, context-aware, and computationally efficient.

\section{Theoretical Framework: The Quantum-Inspired Phase Space}

\subsection{Dynamic Phase Space Representation}

In our model, words are not represented as static vectors; instead, they are treated as dynamic entities existing in a multi-dimensional phase space. Drawing inspiration from quantum mechanics, our phase space representation captures not only static semantic content but also dynamic relationships with their context, emotion, and other nuances. Each word $w$ is represented as a state $|\psi_w\rangle$ in a phase space, where its components capture different dimensions of meaning:
$$
|\psi_w\rangle = \sum_{l=1}^{L} \sum_{i=0}^{d_l-1} \alpha_{l,i} e^{j\phi_{l,i}} |i\rangle_l
$$
Where:
\begin{itemize}
    \item $L$ is the number of layers for different contexts (e.g., semantic, context, emotion).
    \item $d_l$ is the dynamic dimensionality for each layer $l$. Unlike traditional static models, $d_l$ is not fixed and may change based on input complexity.
    \item $\alpha_{l,i}$ is the amplitude component, representing the "strength" of the i-th mode in layer $l$.
    \item $\phi_{l,i}$ is the phase component that captures dynamic relationships and phase shifts.
    \item $|i\rangle_l$ is the i-th basis vector in layer $l$.
\end{itemize}

The dimensionality $d_l$ is not fixed and dynamically changes based on input complexity. When new concepts or layers of meaning are introduced, the space dynamically allocates additional dimensions to represent them. The amplitude components are derived from the embedding layer as well as from the different aspects of the text such as context and emotional content. It is a learned representation during the training phase. These dynamic layers are what make our model unique, and it provides a way to express the fluidity of language. We use bounded trigonometric functions to create stable phase spaces, which means the magnitude of the amplitude component is always within $[-1, 1]$ and the phase component is always within $[-\pi, \pi]$.

\textbf{Example:} Consider the sentence "The cat was very sad." In our dynamic phase space representation:
\begin{itemize}
    \item "cat" will have strong semantic and context dimensions (subject), while emotion might be low.
    \item "sad" will have a strong emotion dimension and a relatively high semantic dimension but lower context dimension.
    \item "was" will have a high context dimension and relatively low for others.
\end{itemize}

\subsection{Universal Constants as Information Scaffolding}

Our model uses fundamental constants like the golden ratio $(\phi = \frac{1 + \sqrt{5}}{2})$, Euler's number $(e)$, and $\pi$, as the foundation for our quantum-inspired system. This approach provides a principled way of creating stable and mathematically robust phase spaces. These constants appear frequently in natural phenomena and mathematics, creating well-structured and stable patterns.
\begin{enumerate}
    \item \textbf{Golden Ratio ($\phi$):} In our model, we use the golden ratio as a fundamental parameter in the phase shifts. The inherent properties of self-similarity and optimal packing of the golden ratio ensure that the phase relationships are both highly diverse and systematically distributed.
        $$
        \phi_{l,i} = \frac{2\pi}{d_l} i \phi
        $$
    \item \textbf{Euler's Number ($e$):} We use Euler's number to ensure that information is smoothly distributed throughout the phase space and to define the bounded behavior of the state transitions.
        $$
        \alpha_{l,i} = e^{-|i|/d_l}
        $$
    \item \textbf{Pi ($\pi$):} Pi defines the fundamental periodicity of our phase space. We use pi in all trigonometric functions to ensure the phase angles are always within a stable and defined range ($[-\pi, \pi]$).
        $$
        \phi_{l,i} = \frac{2\pi}{d_l} i \phi
        $$
\end{enumerate}

By using these constants, our model has mathematically stable phase space and consistent encoding, allowing for a more robust model compared to purely data-driven representations.

\subsection{Layered Approach: Context, Emotion, and More}

We define distinct layers to capture various nuances of language:
\begin{enumerate}
    \item \textbf{Semantic Layer:} Captures the inherent meaning of the word using embeddings ($E_w$). The initial amplitude component is given by,
        $$
        \alpha_{sem,i} = \frac{E_{w,i}}{\sqrt{\sum_{j=0}^{d_{sem}-1} E_{w,j}^2}}
        $$
    \item \textbf{Context Layer:} Captures how words interact and modify each other based on context. The initial amplitude and phase components for context are determined by other nearby tokens.
    \item \textbf{Emotion Layer:} Captures the emotional tone of the word. It is activated based on emotionally charged tokens.
    \item \textbf{Other Layers:} Other layers such as syntactic meaning, temporal meaning, and logical meaning can be added to the model. The addition of new layers depends on the requirements of the specific model and application.
\end{enumerate}

Mathematically, we can formulate these layers by:

For the \textit{semantic layer}, where $E_w$ is the word embedding and $d_{sem}$ is its dynamic dimension, the state is given by:
$$
|\psi_{sem}\rangle = \sum_{i=0}^{d_{sem}-1} \frac{E_{w,i}}{\sqrt{\sum_{j=0}^{d_{sem}-1} E_{w,j}^2}}  e^{j \frac{2\pi}{d_{sem}} i \phi} |i\rangle_{sem}
$$

For the \textit{context layer}, where $c_i$ represents the contextual influence from other tokens, $\phi_{context}$ represents the phase shift from other tokens, and $d_{context}$ is its dynamic dimension, the state is given by:
$$
|\psi_{context}\rangle = \sum_{i=0}^{d_{context}-1} \tanh(c_i)  e^{j (\frac{2\pi}{d_{context}} i \phi + \phi_{context})} |i\rangle_{context}
$$

For the \textit{emotion layer}, where $e_i$ represents the emotional content of the token and $d_{emo}$ is its dynamic dimension, the state is given by:
$$
|\psi_{emo}\rangle = \sum_{i=0}^{d_{emo}-1} \tanh(e_i)  e^{j \frac{2\pi}{d_{emo}} i \phi} |i\rangle_{emo}
$$

These layers interact through interference patterns:
$$
|\psi_{w}\rangle = |\psi_{sem}\rangle + |\psi_{context}\rangle + |\psi_{emo}\rangle + ....
$$
These multiple layers are combined using interference patterns. By summing the different layers, we let the different layers interfere constructively and destructively such that the resultant phase space is a combination of all layers. This gives the model the capability to represent various nuances and richness of the language.

\section{Quantum-Inspired Attention and Information Processing}

\subsection{Interference-Based Attention Mechanism}

Instead of traditional weighted averages, our model uses the \textit{interference patterns} of phase spaces to determine the relationships between words.

Let query $Q$, key $K$, and value $V$ be the phase space representations of words. We compute attention based on the interference between $Q$ and $K$ as follows:

\begin{enumerate}
    \item \textbf{Phase Extraction:} Extract the phases of $Q$ and $K$:
        $$
        \phi_Q = \arctan2(Q_{imag}, Q_{real})
        $$
        $$
        \phi_K = \arctan2(K_{imag}, K_{real})
        $$
    \item \textbf{Phase Difference:} Compute the phase difference $\Delta\phi$:
        $$
        \Delta\phi = \phi_Q - \phi_K
        $$
    \item \textbf{Interference:} Generate the interference pattern using the cosine function for constructive and destructive interference:
        $$
        I = \cos(\Delta\phi)
        $$
    \item \textbf{Attention Scores:} Attention scores are generated using the interference pattern. The model also learns a weight matrix $W_{attn}$ such that the final attention is given by:
        $$
        A = \text{softmax}(I \cdot W_{attn})
        $$
    \item \textbf{Value Transformation:} Transform the value $V$ using the attention scores:
        $$
        \text{Output} = A \cdot V
        $$
\end{enumerate}

The attention output is a combination of interference and the values from other tokens, giving a quantum-based dynamic attention layer. This provides a more holistic and nuanced attention mechanism to capture complex relationships.

\subsection{Non-Local Interactions and Entanglement-like Effects}

Our approach allows non-local interactions, allowing for the capture of long-range dependencies that are not possible with fixed contextual windows. We can achieve this by letting two different phase spaces interact with a global interference layer:
$$
I_{global} =  \sum_{i=1}^{N} \sum_{j=1}^{N} \cos(\phi_i - \phi_j)
$$
Where $i$ and $j$ span over all words in the sentence. This allows for truly non-local interactions as every token will interfere with all the tokens in the sentence. This is different from a traditional attention that has a local scope.
To simulate entanglement-like effects, we synchronize the phase changes between related tokens by using shared phase shifts:
$$
\phi_{i}^{entangled} =  \phi_{i} + \phi_{global}
$$
Where $\phi_{global}$ is the phase change that is being shared across different tokens. This entangles two or more different tokens based on global properties. This allows related tokens to change their phases together and thus create interdependencies throughout the text. These non-local interactions and entanglement effects capture long-range dependencies better and also create a more holistic and robust understanding of language.

\subsection{Bounded State Evolution}

To ensure numerical stability, we use bounded operations such as $\tanh$ and $\sin$ for the evolution of states. For the amplitude component, we use:
$$
\alpha' = \tanh(\alpha)
$$
and for the phase component, we use:
$$
\phi' = \sin(\phi)
$$
These transformations ensure that the state magnitude is always between [-1, 1] and phase angles are between $[-\pi, \pi]$, and also stabilize gradient flows, and give a much stable and efficient training.

\section{Concept-Based Tokenization and Language Agnostic Representations}

\subsection{Concept Layer Representation}

In our concept-based tokenization, words are viewed as instances of underlying concepts that are more abstract than the words themselves. Each word’s state $|\psi_w\rangle$ is mapped to a corresponding concept state $|\psi_c\rangle$ such that:
$$
|\psi_c\rangle = F(|\psi_w\rangle)
$$
where $F$ is a transformation that extracts and transforms the core concept. The concept is represented in a dynamic phase space, similar to the word phase space. However, the concept phase space is designed to capture the abstract idea behind the word rather than the specific linguistic instance. For example, "dog", "perro", and "chien" can all be mapped to a similar concept in the concept phase space using this transformation.

\subsection{Concept-Word Mapping and Multi-Lingual Representation}

Each concept can be mapped to multiple words. We learn transformations $M_l$ between the concept phase space and the word phase spaces, where $l$ is the target language:
$$
|\psi_{w,l}\rangle = M_l(|\psi_c\rangle)
$$
$M_l$ is a language-specific transformation matrix, which is learned during training. This matrix transforms a concept into the correct word representation for the target language. By learning different transformations, the model can generate a single concept to words in multiple languages. The concept-based representation allows for cross-lingual learning, where the model can learn language-invariant features in the concept layer. By training the model with different languages at the same time, the shared concept layer enhances the model’s ability to understand and generate language from multiple language datasets.

\subsection{Concept-Based Generation for Multiple Languages}

Text generation is a two-stage process. In the first stage, the model operates in the concept space, generating a sequence of concepts. In the second stage, these concepts are mapped to word sequences using language-specific transformation matrices ($M_l$). For a language $l$:
$$
\text{text}_l = M_l ( \text{concepts})
$$
This allows the same model to generate text in multiple languages. By using a shared concept space, we reduce the need to store language-specific parameters. Therefore, our model will be able to generalize to many languages.

\section{Training and Loss Functions}

\subsection{Unitary-Inspired Loss Functions}

To ensure stable state evolution, we use loss functions that encourage unitary behavior. A unitary transformation, $U$, preserves the length of vectors ($||Uv|| = ||v||$), and we need our state transitions to be unitary-like. Therefore, we can introduce a loss function that penalizes deviations from unitary behavior. We can define this as:
$$
Loss_{unitary} = || ||U( |\psi\rangle) ||^2 - || |\psi\rangle ||^2 ||
$$
This loss penalizes any changes to state magnitude and prevents divergence. Another loss function to encourage unitary behavior can be the orthogonality of different basis states:
$$
Loss_{ortho} = ||\langle i| j \rangle - \delta_{ij} ||
$$
where $\delta_{ij}$ is the Kronecker delta. These loss functions ensure that the state transitions mimic unitary behavior, which is vital for maintaining stable and converging training.

\subsection{Phase Coherence Maximization}

Phase coherence measures how consistently phases are aligned among related tokens. This maximizes useful interference and minimizes destructive interference. We quantify phase coherence between tokens as:
$$
Coherence = \frac{1}{N} |\sum_{i=1}^{N} e^{j \phi_i} |
$$
We maximize this coherence for tokens that are contextually relevant to each other. This forces the model to learn a representation that brings related tokens in phase such that they constructively interfere.

\subsection{Energy-Based Training and Regularization}

We frame model training as finding a minimal energy state, where energy is based on interference patterns and phase coherence. We use interference patterns for implicit regularization, where coherent phases will have low energy and incoherent ones will have high energy. The energy function for the model can be defined as:
$$
Energy = -\frac{1}{N} \sum_{i=1}^{N} \sum_{j=1}^{N} \cos(\phi_i - \phi_j) - \alpha \text{Coherence}
$$
where $\alpha$ is a hyper parameter. By minimizing energy, the model also regularizes itself using interference and by maximizing phase coherence. By defining training using energy, we force the model to reach a stable state that also promotes stable output and learning.

\section{Implementation Details}

\subsection{Module Design}
Our architecture is implemented as a series of independent modules:

\begin{enumerate}
    \item \textbf{Phase Space Encoder:} Transforms input tokens into a dynamic phase space representation based on multiple layers. This encoder also prepares phase and amplitude components for use in other modules.
        \begin{lstlisting}[caption=Phase Space Encoder Module]
class PhaseSpaceEncoder(nn.Module):
    def __init__(self, vocab_size, d_model, n_layers, constants):
        super().__init__()
        self.embedding = nn.Embedding(vocab_size, d_model)
        self.d_model = d_model
        self.constants = constants
        self.n_layers = n_layers

    def forward(self, x):
        B, L = x.shape
        embeddings = self.embedding(x)
        states = torch.zeros(B, L, self.d_model * self.n_layers, dtype=torch.complex64)
        # Apply phase encoding for different layers
        for b in range(B):
            for l in range(L):
                for i in range(self.n_layers):
                     # Generate phase spaces based on embedding, context etc.
                    alpha_i = torch.tanh(embeddings[b, l]) # using tanh for bounded magnitude
                    phi_i = torch.sin(torch.linspace(0, 2 * np.pi, self.d_model) * self.constants.phi) # using sin for bounded phases
                    states[b, l] = torch.cat((alpha_i * torch.exp(1j * phi_i), states[b, l]), dim=0)
        return states
        \end{lstlisting}

    \item \textbf{Interference Attention Module:} Computes attention based on the interference patterns between states.
        \begin{lstlisting}[caption=Interference Attention Module]
class InterferenceAttention(nn.Module):
    def __init__(self, d_model, constants):
        super().__init__()
        self.d_model = d_model
        self.constants = constants
        self.attn_weights = nn.Linear(d_model, d_model)

    def forward(self, q, k, v):
        q_phase = torch.atan2(q.imag, q.real)
        k_phase = torch.atan2(k.imag, k.real)
        delta_phi = q_phase - k_phase
        interference = torch.cos(delta_phi) # use cosine for constructive and destructive interference
        attn_scores = self.attn_weights(interference) # learnable weights
        attn_weights = torch.softmax(attn_scores, dim=-1)
        output = torch.matmul(attn_weights, v)
        return output
        \end{lstlisting}
    \item \textbf{Concept Layer:} Implements concept-based tokenization and manages concept-word mapping.
        \begin{lstlisting}[caption=Concept Layer Module]
class ConceptLayer(nn.Module):
    def __init__(self, vocab_size, d_model, num_concept):
        super().__init__()
        self.vocab_size = vocab_size
        self.num_concept = num_concept
        self.concept_embedding = nn.Embedding(num_concept, d_model)
        self.lang_map = nn.ModuleDict({
          "en": nn.Linear(d_model, d_model),
          "es": nn.Linear(d_model, d_model)
        })

    def concept_map(self, w):
        # use embedding to map word into concept space
        concept_id = self.embedding(w)
        return concept_id

    def language_map(self, concept_id, lang):
        # use language specific matrix to transform into word space
        mapped = self.lang_map[lang](concept_id)
        return mapped

    def forward(self, x, lang):
        # extract and transform the core concept to word space based on language
        concept = self.concept_map(x)
        word = self.language_map(concept, lang)
        return word
        \end{lstlisting}
    \item \textbf{Training Module:} Implements loss functions and performs a single training step with gradient clipping.
        \begin{lstlisting}[caption=Training Module]
class TrainingModule():
    def __init__(self, loss_fn, optimizer, clip_grad):
        self.loss_fn = loss_fn
        self.optimizer = optimizer
        self.clip_grad = clip_grad

    def training_step(self, model, batch):
        self.optimizer.zero_grad()
        output = model(batch)
        loss = self.loss_fn(output, batch)
        loss.backward()
        torch.nn.utils.clip_grad_norm_(model.parameters(), self.clip_grad)
        self.optimizer.step()
        return loss.item()
        \end{lstlisting}
\end{enumerate}

PyTorch provides a convenient framework for implementing the proposed model, leveraging its optimized tensor operations and automatic differentiation capabilities. However, the underlying principles, from dynamic dimensionality and phase space representation to interference-based attention and unitary-inspired training, are distinct from conventional machine learning approaches.

\subsection{Real-Valued Approximations and Hardware Considerations}

To work with current GPUs, which are optimized for real-valued computations, we use the following approximations for complex operations:
\begin{enumerate}
    \item \textbf{Phase Mapping:} We map complex numbers into real space using trigonometric functions:
        $$
        z = r e^{j\theta} \rightarrow [r\cos(\theta), r\sin(\theta)]
        $$
    We use $\sin$ and $\cos$ function to transform phases into real values.
    \item \textbf{Interference Simulation:} We simulate interference by using the cosine function on the phase difference rather than complex number multiplications; this achieves the same result in the real domain.
    \item \textbf{State Encoding:} We encode quantum states into real-valued vectors using $\sin$ and $\tanh$ functions for bounded behavior.
    \item \textbf{Matrix Manipulation:} All the matrix multiplications are done in the real domain.
    \item \textbf{Pre-Computed Values:} We compute values such as $\sin$ and $\cos$ in advance and use lookup tables. We cache these values to reduce redundant computations.
\end{enumerate}

\subsection{Memory Management}

We reduce memory usage through a multi-pronged approach:
\begin{enumerate}
    \item \textbf{Dynamic Dimension:} The dynamic nature of phase spaces reduces the need to allocate excess memory for static dimensions.
    \item \textbf{Caching:} Values for the trigonometric functions, interference patterns, and concept-word mapping are cached and reused for computation, reducing redundant calculations.
    \item \textbf{Memory Clearing:} Memory is cleared after each training step by removing intermediary variables to prevent memory leaks.
    \item \textbf{Chunking:} Long sequences are processed by breaking into chunks, thereby reducing the memory footprint.
\end{enumerate}

\section{Preliminary Experiments and Analysis}
Our initial experiments show that the model is able to converge effectively when trained on synthetic data:
\begin{enumerate}
    \item \textbf{Training:} Training losses reduce steadily, indicating that our model learns effectively.
    \item \textbf{Stability:} Training was stable with no exploding gradients or vanishing gradients and by optimizing the learning rate.
    \item \textbf{Text Generation:} The model was able to produce coherent text with some variations in phase.
    \item \textbf{Phase Coherence:} The phase coherence function maximizes and stabilizes indicating that the model is using interference for learning.
\end{enumerate}
Note: We acknowledge that these experiments are basic and require more rigorous experimentation and benchmark evaluation in the future.

\section{Discussion}

Our proposed framework offers a novel approach to language modeling compared to traditional transformers:
\begin{enumerate}
    \item \textbf{Dynamic Phase Space:} Unlike static vectors, our dynamic phase space represents words as multi-faceted entities that evolve with context.
    \item \textbf{Interference Attention:} Our interference-based attention is more context-sensitive and better at capturing long-range dependencies and relationships.
    \item \textbf{Concept-Based Tokenizer:} Our model learns language-agnostic concept representation and gives the ability to transfer knowledge across languages.
    \item \textbf{Training and Loss Functions:} We use loss functions that encourage stable and coherent learning by using unitary operations and phase coherence.
    \item \textbf{Memory Efficiency:} Our memory optimization allows the model to be less memory intensive and also makes it easy to run on standard GPUs.
\end{enumerate}

\textbf{Limitations:} This is a theoretical framework and the implementation is not perfect. The current model lacks standard evaluations in language benchmarks, and more work is required to achieve state-of-the-art results in such tasks.

\section{Conclusion}

This paper presents a quantum-inspired language model that offers a fresh approach to NLP by moving beyond the constraints of traditional neural networks. The model leverages dynamic phase spaces, interference-based attention, concept-based tokenization, and unitary-inspired training. The approach provides a modular design, a math-backed theory, and a practical implementation path. Our goal is to create more adaptive and context-aware models that are also robust, stable, and less memory-intensive. This work lays the foundation for future research directions by creating novel ways for understanding and modeling languages.

\section*{Appendix A: Mathematical Formulations}

\begin{itemize}
    \item \textbf{Dynamic Phase Space Representation:}
        $$
        |\psi_w\rangle = \sum_{l=1}^{L} \sum_{i=0}^{d_l-1} \alpha_{l,i} e^{j\phi_{l,i}} |i\rangle_l
        $$
    \item \textbf{Interference-Based Attention:}
        $$
        A(Q,K,V) = \text{softmax}( \cos(\arctan2(Q_{imag}, Q_{real}) - \arctan2(K_{imag}, K_{real})) \cdot W_{attn})V
        $$
    \item \textbf{Concept-Word Mapping:}
        $$
        |\psi_{w,l}\rangle = M_l(|\psi_c\rangle)
        $$
    \item \textbf{Unitary Inspired Loss:}
        $$
        Loss_{unitary} = || ||U( |\psi\rangle) ||^2 - || |\psi\rangle ||^2 ||
        $$
        $$
         Loss_{ortho} = ||\langle i| j \rangle - \delta_{ij} ||
        $$
    \item \textbf{Phase Coherence:}
        $$
        Coherence = \frac{1}{N} |\sum_{i=1}^{N} e^{j \phi_i} |
        $$
    \item \textbf{Energy-Based Loss:}
        $$
        Energy = -\frac{1}{N} \sum_{i=1}^{N} \sum_{j=1}^{N} \cos(\phi_i - \phi_j) - \alpha \text{Coherence}
        $$
\end{itemize}

\section*{Appendix B: Code Implementation}
The complete code for each module, along with detailed comments, will be released in the supplementary material soon.

\end{document}